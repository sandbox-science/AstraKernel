% ------------------------------------------------
% *** Preface ***
% ------------------------------------------------
\doublespacing
\newpage
\addcontentsline{toc}{section}{Preface}
\section*{Preface}

AstraKernel is a minimal, experimental operating system kernel written in modern 
C and ARM assembly. Designed to run on QEMU’s VersatilePB (ARM926EJ-S) emulated 
platform, AstraKernel serves as a practical and approachable foundation for 
learning and experimenting with core operating system concepts.

This project was developed with a focus on clarity, simplicity, and educational 
value. Rather than attempting to re-create the complexity of established operating 
systems, AstraKernel goal is to strip away unnecessary abstractions and present a 
clean, understandable codebase for anyone interested in the "bare metal" 
foundations of computing.

Through hands-on implementation of kernel bootstrapping, direct hardware 
communication, and basic user interaction, AstraKernel demonstrates how 
fundamental OS components come together. The project showcases how modern C 
best practices can be utilized in a systems programming context to create 
code that is maintainable, portable, and robust while still being
accessible to those new to kernel development. The design of AstraKernel 
emphasizes modularity and extensibility, allowing developers to easily add new 
features or modify existing ones. This makes it ideal for educational purposes, 
as it provides a clear structure that can be followed and built upon.

It is my hope that AstraKernel will not only serve as a stepping stone 
for those wishing to understand kernel development, but also inspire curiosity 
and confidence in exploring lower-level aspects of computer systems.

