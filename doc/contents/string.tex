\newpage
\section{String Manipulation API}
\subsection{\texttt{strcmp(const char *str\_1, const char *str\_2)}}

\paragraph{Purpose}
Compares two null-terminated strings, character by character.

\paragraph{Overview}
This function compares the characters of two strings (\texttt{ASCII values}), 
\texttt{str\_1} and \texttt{str\_2}, one by one. It returns:

\paragraph{Return Values}
\begin{itemize}
    \item 0 if both strings are equal,
    \item -1 if the first differing character in \texttt{str\_1} is less than that in \texttt{str\_2},
    \item 1 if the first differing character in \texttt{str\_1} is greater than that in \texttt{str\_2}.
\end{itemize}

\paragraph{Behavior}
The comparison stops at the first difference or the null terminator. 
Case sensitivity is observed. Behavior is undefined if either pointer is not a 
valid null-terminated string.

\subsection*{Examples}
\begin{lstlisting}[language=C, caption=String Comparison Example]
  int result = strcmp("abc", "abc"); // Expect 0
  printf("Expect 0 -> %d\n", result);

  result = strcmp("abc", "abd"); // Expect -1
  printf("Expect -1 -> %d\n", result);

  result = strcmp("abc", "ABC"); // Expect 1
  printf("Expect 1 -> %d\n", result);
\end{lstlisting}

\begin{note}
In the future, we could returning the difference between the first differing 
characters, which would allow for more detailed comparisons. This would enable 
users to understand how far apart the strings are in terms of character values.
\end{note}

\subsection{\texttt{strlen(const char *str)}}

\paragraph{Purpose}
Calculates the length of a null-terminated string.

\paragraph{Overview}
This function counts the number of characters in a null-terminated string,
excluding the null terminator itself. It returns the length of the string as an
unsigned integer.

\paragraph{Return Values}
\begin{itemize}
    \item The number of characters in the string, not including the null terminator.
    \item 0 if the string is empty (i.e., the first character is the null terminator).
\end{itemize}

\paragraph{Behavior}
The function iterates through the input string until the null terminator is found, returning 
the number of characters preceding it. If the input pointer is not a valid 
null-terminated string, the behavior is undefined.

\subsection*{Examples}
\begin{lstlisting}[language=C, caption=String Length Example]
  size_t result = strlen("abc"); // Expect 3
  printf("Expect len 3 -> %d\n", result);

  result = strlen(""); // Expect 0
  printf("Expect len 0 -> %d\n", result);
\end{lstlisting}
