% ------------------------------------------------
% *** Section 2: I/O \& Time Services ***
% ------------------------------------------------
\newpage
\chapter{I/O \& Time Services}

\section{UART Output API}

\subsubsection{Registers \& Constants}
\begin{itemize}
  \item \textbf{UART0\_DR} (Data Register), \texttt{0x101f1000}
  \item \textbf{UART0\_FR} (Flag Register), \texttt{0x101f1018}
  \item \textbf{UART\_FR\_TXFF}, \textbf{UART\_FR\_RXFE}
\end{itemize}

\subsection{\texttt{printf(char *s, ...)}}

\paragraph{Description}
Sends a null-terminated format string over UART. If an incorrect datatype is given 
for a format specifier, \underline{the behavior is undefined}. If a format specifier 
is given without a matching argument, it is simply skipped when the string is outputted. 
The following format specifiers are supported:

\paragraph{Supported Format Specifiers}
\begin{itemize}
  \item \texttt{\%c}: Expects a single character.
  \item \texttt{\%s}: Expects a null-terminated string.
  \item \texttt{\%d}: For 32-bit signed integers in the range \texttt{-2147483648} to \texttt{2147483647}.
  \item \texttt{\%ld}: For 64-bit signed integers in the range \texttt{-9223372036854775808} to \texttt{9223372036854775807}, excluding the range specified above, for \textbf{\%d}.
  \item \texttt{\%lu}: For 64-bit unsigned integers in the range \texttt{2147483648} to \texttt{18446744073709551615}.
  \item \texttt{\%x}, \texttt{\%X}: For 32-bit unsigned integers in the range \texttt{0} to \texttt{2147483647}, where the case of \texttt{x} determines the case of the hexadecimal digits (\texttt{a-f} or \texttt{A-F}).
  \item \texttt{\%lx}, \texttt{\%lX}: For 64-bit unsigned integers printed in hexadecimal format. Has the same range as \textbf{\%lu}. The case of \texttt{x} determines the case of the hexadecimal digits (\texttt{a-f} or \texttt{A-F}).
  \item \texttt{\%\%}: Outputs a literal \texttt{\%}.
  \item \texttt{\%}: If the format specifier is not followed by a valid character, prints \texttt{(null)}.
\end{itemize}

\begin{flushleft}
All the ranges specified above are inclusive. Also, note that integers in the 
range \texttt{-2147483648} to \texttt{2147483647} are passed as 32 bit integers 
and any integers not part of this range are passed as 64 bit integers \underline{by default}.
If desired, integers of this range can be cast as \texttt{long long} or \texttt{unsigned long long} 
for use with the format specifiers prefixed by the character \texttt{l}.
\end{flushleft}

\subsection*{Examples}
\begin{lstlisting}[language={C}, caption={Examples of printf usage in AstraKernel.}, label={lst:printf_examples}]
  // Printing long long integers
  printf("%lu %ld %ld\n", 18446744073709551615, -9223372036854775808, 
  9223372036854775807);

  // Printing 32-bit signed integers
  printf("%d %d\n", 2147483647, -2147483648);

  // Printing 32-bit unsigned integers
  printf("%x %x %X %X\n", 2147483647, 1234, 2147483647, 1234);
  // Output: 7fffffff 4d2 7FFFFFFF 4D2

  // Printing unsigned long long integers in hex
  printf("%lX %lx\n", 0x123456789ABCDEF0, 9223372036854775809);
  // Output: 123456789ABCDEF0 8000000000000001

  // Printing a character
  printf("Name: %c\n", 'b');

  // Printing a string
  printf("Hello %s\n", "World");

  // Printing a '%'
  printf("100%%\n");

  // Printing a null character
  printf("%\n");
  // Output: (null)
\end{lstlisting}
