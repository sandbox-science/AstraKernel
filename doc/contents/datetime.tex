\newpage
\section{Timekeeping API}
Provide calendar dates and clock time based on the on-chip RTC.

\paragraph{Overview}
\begin{itemize}
    \item Reads a 32-bit seconds-since-1970 counter (RTC register at `0x101e8000`).
    \item Exposes two APIs:
    \begin{itemize}
        \item \texttt{getdate(dateval *date)}: Fetches current date in days, months, and years.
        \item \texttt{gettime(timeval *time)}: Fetches current time in hours, minutes, and seconds.
    \end{itemize}
\end{itemize}

\subsection*{Data Structure}

\begin{lstlisting}[language=C, caption={Structure definitions for time and date values.}, label={lst:datetime_structs}]
    typedef struct
    {
        uint32_t hrs;
        uint32_t mins;
        uint32_t secs;
    } timeval;

    typedef struct
    {
        uint32_t day;
        uint32_t month;
        uint32_t year;
    } dateval;
\end{lstlisting}

\noindent
The number of seconds since epoch is retrieved from 926EJ-S's Real Time Clock 
Data Register(\textbf{RTCDR}). This is a 32 bit register, and therefore 
\underline{subject to the Year 2038 Problem}.

\subsection{\texttt{uint32\_t getdate(dateval *date\_struct)}}
Populate the provided \texttt{*date\_struct} with the current date in days, months, and years.

\paragraph{Parameters}
\begin{itemize}
    \item \texttt{date\_struct} pointer to a \texttt{dateval} structure; if \texttt{NULL}, 
    only the raw \textbf{RTC} counter is returned.
\end{itemize}

\paragraph{Return Value}
\begin{itemize}
    \item Returns number of seconds since epoch (\texttt{RTCDR}) as read from the hardware register.
\end{itemize}

\subsection{\texttt{uint32\_t gettime(timeval *time\_struct)}}
Populate \texttt{*time\_struct} with the current click time in hours, minutes, and seconds.

\paragraph{Parameters}
\begin{itemize}
    \item \texttt{time\_struct} pointer to a \texttt{timeval} structure; if \texttt{NULL}, 
    only the raw \textbf{RTC} counter is returned.
\end{itemize}

\paragraph{Return Value}
\begin{itemize}
    \item Returns number of seconds since epoch (\texttt{RTCDR}) as read from the hardware register.
\end{itemize}

\subsection*{Examples}

\begin{lstlisting}[language=C, caption={Examples of getdate and gettime usage in AstraKernel.}, label={lst:datetime_examples}]
    gettime(&time_struct);
    printf("Current time(GMT): %d:%d:%d\n", time_struct.hrs, time_struct.mins, time_struct.secs);

    getdate(&date_struct);
    printf("Current date(MM-DD-YYYY): %d-%d-%d\n", date_struct.month, date_struct.day, date_struct.year);
\end{lstlisting}

\subsection*{Notes}
\begin{itemize}
    \item Assumes \texttt{uint32\_t} is 4-bytes (compile-time check via \texttt{\_Static\_assert}).
    \item The time and date values are based on the system's RTC, which should be set 
    correctly at boot time.
    \item The API does not handle time zones or daylight saving time adjustments; it 
    provides GMT time only.
\end{itemize}