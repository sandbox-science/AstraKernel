% ------------------------------------------------
% *** Section: Error Handling ***
% ------------------------------------------------
\newpage
\chapter{Error Handling}
\section{Error Codes}
\paragraph{Overview}
AstraKernel uses a small set of negative error codes defined in
\texttt{include/errno.h}. These values are intended for internal kernel APIs
that may fail during early boot or runtime.

\paragraph{Defined Codes}
\begin{itemize}
  \item \texttt{KERR\_OK} (0): success.
  \item \texttt{KERR\_NOT\_FOUND} (-1): resource not found.
  \item \texttt{KERR\_NOMEM} (-2): out of memory.
  \item \texttt{KERR\_NO\_SPACE} (-3): no space available.
  \item \texttt{KERR\_INVAL} (-4): invalid request or parameter.
\end{itemize}

\section{Helpers}
\paragraph{Overview}
The errno module provides small helper functions for checking and displaying
error codes:

\begin{itemize}
  \item \texttt{kerr\_is\_ok(e)} returns true if \texttt{e == KERR\_OK}.
  \item \texttt{kerr\_is\_err(e)} returns true if \texttt{e < 0}.
  \item \texttt{error\_str(e)} returns a static string for the code.
\end{itemize}

\subsection*{Example}
\begin{lstlisting}[language=C, caption={Error code formatting example.}]
  kerror_t err = KERR_NOMEM;
  if (kerr_is_err(err))
  {
      printf("error: %s\n", error_str(err));
  }
\end{lstlisting}
