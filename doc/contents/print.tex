% ------------------------------------------------
% *** Section 2: I/O \& Time Services ***
% ------------------------------------------------
\newpage
\chapter{I/O \& Time Services}

\section{UART Output API}

\subsubsection{Registers \& Constants}
\begin{itemize}
  \item \textbf{UART0\_DR} (Data Register), \texttt{0x101f1000}
  \item \textbf{UART0\_FR} (Flag Register), \texttt{0x101f1018}
  \item \textbf{UART\_FR\_TXFF}, \textbf{UART\_FR\_RXFE}
\end{itemize}

\subsection{\texttt{printf(char *s, ...)}}

\paragraph{Description}
Sends a null-terminated format string over UART. If an incorrect datatype is given
for a format specifier, \underline{the behavior is undefined}. If a format specifier
is given without a matching argument, the behavior is undefined. The following
format specifiers are supported:

\paragraph{Supported Format Specifiers}
\begin{itemize}
  \item \texttt{\%c}: Expects a single character.
  \item \texttt{\%s}: Expects a null-terminated string.
  \item \texttt{\%d}: Signed integers (\texttt{int}).
  \item \texttt{\%u}: Unsigned integers (\texttt{unsigned int}).
  \item \texttt{\%x}, \texttt{\%X}: Unsigned integers printed in hexadecimal; case depends on \texttt{x}/\texttt{X}.
  \item \texttt{\%ld}: Signed \texttt{long} integers.
  \item \texttt{\%lu}: Unsigned \texttt{long} integers.
  \item \texttt{\%lx}, \texttt{\%lX}: Unsigned \texttt{long} printed in hexadecimal.
  \item \texttt{\%p}: Pointer, printed as \texttt{0x} followed by hex digits.
  \item \texttt{\%\%}: Outputs a literal \texttt{\%}.
  \item Unknown specifier: the sequence is printed verbatim as \texttt{\%x} (where \texttt{x} is the unknown specifier).
\end{itemize}

\begin{flushleft}
On ARMv7-A, \texttt{int} and \texttt{long} are 32-bit, so \texttt{\%d} and \texttt{\%ld}
are the same width (likewise for \texttt{\%u} and \texttt{\%lu}). 64-bit integer
formatting (e.g., \texttt{\%lld} or \texttt{\%llu}) is not implemented.
\end{flushleft}

\subsection*{Examples}
\begin{lstlisting}[language={C}, caption={Examples of printf usage in AstraKernel.}, label={lst:printf_examples}]
  // Printing 32-bit signed integers
  printf("%d %d\n", 2147483647, -2147483648);

  // Printing 32-bit unsigned integers
  printf("%x %x %X %X\n", 2147483647, 1234, 2147483647, 1234);
  // Output: 7fffffff 4d2 7FFFFFFF 4D2

  // Printing pointers
  printf("UART base: %p\n", (void *)0x101F1000);

  // Printing a character
  printf("Name: %c\n", 'b');

  // Printing a string
  printf("Hello %s\n", "World");

  // Printing a '%'
  printf("100%%\n");

  // Unknown specifiers are passed through
  printf("%q\n");
  // Output: %q
\end{lstlisting}
