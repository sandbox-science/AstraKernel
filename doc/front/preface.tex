% ------------------------------------------------
% *** Preface ***
% ------------------------------------------------
\onehalfspacing
\newpage
\addcontentsline{toc}{section}{Preface}
\section*{Preface}

This documentation serves as a comprehensive guide to the AstraKernel project, 
a minimal operating system kernel written in modern C and ARM assembly. 
Designed to run on QEMU’s VersatilePB (ARM926EJ-S) emulated platform, 
AstraKernel is intended to provide a clear and approachable introduction 
to the fundamental concepts of operating system design and development.
This project also reflects my personal journey in learning about kernel development  
and systems programming.

This project was developed with a focus on clarity, simplicity, and educational value. 
Rather than attempting to recreate the complexity of established operating systems, 
AstraKernel’s goal is to strip away unnecessary abstractions and present a clean, 
understandable codebase for anyone interested in the "bare metal" foundations of computing.

Through hands-on implementation of kernel bootstrapping, direct hardware communication, 
and basic user interaction, AstraKernel demonstrates how fundamental OS components 
come together. The project showcases how modern C best practices can be utilized 
in a systems programming context to create code that is maintainable, portable, and robust, 
while still being accessible to those new to kernel development. The design of this kernel 
emphasizes modularity and extensibility, allowing developers to easily add new features 
or modify existing ones. This makes it ideal for educational purposes, as it provides 
a clear structure that can be followed and built upon.

It is my hope that AstraKernel will not only serve as a foundation for those wishing 
to understand kernel development, but also inspire curiosity and confidence in exploring 
lower-level aspects of computer systems.

\begin{info}
  This documentation is a work in progress and may be updated
  as the project evolves. I welcome contributions, feedback,
  and suggestions for improvement. You can find the source code on GitHub:
  \url{https://github.com/sandbox-science/AstraKernel}
\end{info}

\addcontentsline{toc}{section}{About This Project}
\section*{About This Project}
\paragraph{Resources}

To guide my learning and support development of AstraKernel, I am using the following resources, 
which are particularly valuable for foundational and practical understanding of OS design:
\begin{itemize}
  \item \textbf{Operating Systems: Three Easy Pieces} by Remzi H. Arpaci-Dusseau and Andrea C. Arpaci-Dusseau
  \item \textbf{The Little Book About OS Development} by Erik Helin and Adam Renberg
\end{itemize}

\paragraph{Contributions}

AstraKernel is an open source project, and I encourage contributions from anyone 
interested in improving, extending the kernel or simply experiment with it. I also 
encourage anyone to improve this documentation, as it is a work in progress.
If you would like to contribute, please feel free to open an issue or pull request on the GitHub repository.
\\
\begin{info}
  You can join the Matrix Server \url{https://matrix.to/#/#sandboxscience:matrix.org} 
  for general SandBox Science discussion as well as project-specific discussion such 
  as AstraKernel development.
\end{info}

\paragraph{Acknowledgments}
Specially thanks to the following individuals for their contributions:
\begin{itemize}
  \item \textbf{shade5144} (\url{https://github.com/shade5144}) for contributing 
  the initial version of the \texttt{printf} and \texttt{datetime} function as 
  well as the documentation for them.
\end{itemize}

\paragraph{Disclaimer}

AstraKernel is currently in its early stages of development and is not intended for production use. 
It is primarily an educational and experimental project. The code is provided as-is, 
without any warranties or guarantees of any kind. Use it at your own risk.
